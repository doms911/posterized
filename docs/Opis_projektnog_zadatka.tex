\chapter{Opis projektnog zadatka}
		
		Razvijamo aplikaciju namijenjenu sudionicima stručne konferencije, s ciljem pojednostavljenja pregleda i ocjenjivanja radova. Naša misija je kreirati integrirano digitalno okruženje koje potiče interakciju između sudionika i autora te promiče znanstveni i stručni dijalog. U skladu s tim, implementiramo ključne korisničke zahtjeve kako bismo osigurali optimalno korisničko iskustvo i funkcionalnost sustava.
		
		Potrebno je izgraditi sustav za hotelijere (naručitelje) kojim bi se održane konferencije mogle pratiti putem aplikacije. Aplikacija je sustav preko kojeg se mogu pregledavati sadržaji poput postera i radova autora. Svaki autor može prijaviti svoj rad i administratoru konferencije proslijediti materijale potrebne za stavljanje svog rada u aplikaciju. Za svaki pojedini rad, korisnik ima priliku glasati u cilju biranja 3 najbolja rada. Tijekom održavanja konferencije, na aplikaciji postoji video prijenos u stvarnom vremenu te konferencije. Na kraju održavanja konferencije svaki korisnik može preuzeti fotografije s konferencije.
		
		Ključni zahtjev je omogućiti istovremeni rad više korisnika u stvarnom vremenu. Naša aplikacija mora podržavati istovremene aktivnosti korisnika bez gubitka performansi. Na primjer, dok jedan sudionik pregledava rad, drugi može istovremeno davati svoje ocjene ili komentare. Ova funkcionalnost je ključna za dinamičnu interakciju i suradnju među sudionicima konferencije.
		
		Još jedan važan zahtjev je brz pristup bazi podataka. Kada korisnici pristupe aplikaciji za pretraživanje radova ili unos svojih ocjena, odziv sustava mora biti brz i učinkovit. Ovaj zahtjev osigurava da korisnici ne doživljavaju frustracije zbog dugotrajnog čekanja na učitavanje informacija, što je posebno važno u okruženju konferencije gdje je vrijeme sudionika dragocjeno.
		
		Naša aplikacija ima sigurnu, brzu i pouzdanu komunikaciju s bazom podataka i stabilnost veze sa serverom. Pružamo zaštitu povjerljivih podataka konferencije i korisnika te osiguravanje neprekidnog rada aplikacije.
		
		\newpage
		
		Sustav je prilagođen za rad na mobilnim uređajima. Aplikacija je prilagodljiva i lako koristiva na različitim veličinama ekrana, od pametnih telefona do tableta. Sudionicima se omogućava da pristupe i koriste aplikaciju bilo kada i bilo gdje, što je posebno važno u današnjem mobilnom i povezanom svijetu.
		
		Svi ovi zahtjevi su usmjereni na stvaranje korisničkog iskustva koje je učinkovito, intuitivno i prilagođeno potrebama sudionika stručne konferencije. Cilj nam je osigurati da aplikacija ispuni očekivanja korisnika pružajući inovativnu platformu za razmjenu znanstvenih i stručnih informacija.
		
		Brojne su koristi ovog projekta, a one kojima naš sustav pridonosi su: 
		
		\begin{packed_item}
			\item Povećana Interaktivnost: Omogućuje aktivnije sudjelovanje posjetitelja kroz glasovanje i povratne informacije.
			\item Bolja Organizacija: Automatizira prijavu radova i administrativne procese, štedeći vrijeme i resurse.
			\item Pristupačnost: Digitalizacija sadržaja konferencije osigurava lakši pristup informacijama za sve sudionike.
			\item Ekološka Održivost: Smanjenje upotrebe papira kroz digitalne postere i materijale.
			\item Analitički Podaci: Skupljanje podataka o preferencijama sudionika, korisnih za buduće događaje.
		\end{packed_item}

		Postoje razna slična rješenja ovakvih platformu poput Whova, EventMobi, i Attendify. Međutim, naša aplikacija se razlikuje specifičnim funkcijama poput prilagođenog glasovanja, notifikacija i integracije s lokalnim vremenskim uvjetima, što je prilagođeno specifičnim potrebama naše ciljane konferencije.
		
		Korisnici ove aplikacije i zahtjevi svakog korisnika:
		\begin{packed_item}
			\item Posjetitelji Konferencije: Željni pristupa sadržaju i interakcije.
			\item Autori Radova: Traže platformu za predstavljanje rada i dobivanje povratnih informacija.
			\item Organizatori Konferencija: Teže efikasnom upravljanju sadržajem i interakcijom sudionika.
			\item Sponzori: Žele promovirati svoje brendove putem aplikacije.
		\end{packed_item}
		Aplikacija će biti dizajnirana modularno, omogućujući prilagodbu za različite tipove konferencija, broj sudionika, i specifične zahtjeve organizatora. Projekt uključuje razvoj i implementaciju aplikacije, testiranje funkcionalnosti, te suradnju s krajnjim korisnicima za povratne informacije i unaprjeđenja.
		
		\newpage
		
		Postoje razne nadogradnje našeg projektnog zadatka, a neka od njih su sinteza sustava s ovim tehnologijama:
		\begin{packed_item}
			\item Umjetna Inteligencija: Personalizirani prijedlozi sesija bazirani na interesima korisnika.
			\item Virtualna Stvarnost: Mogućnost virtualnog obilaska konferencije za korisnike koji ne mogu prisustvovati fizički.
			\item Integracija s Društvenim Mrežama: Olakšavanje dijeljenja sadržaja i povećanje vidljivosti konferencije.
		\end{packed_item}

		
		
		\eject
		
		
	