\chapter{Dnevnik promjena dokumentacije}
		
				
		
		\begin{longtblr}[
				label=none
			]{
				width = \textwidth, 
				colspec={|X[2]|X[13]|X[3]|X[3]|}, 
				rowhead = 1
			}
			\hline
			\textbf{Rev.}	& \textbf{Opis promjene/dodatka} & \textbf{Autori} & \textbf{Datum}\\[3pt] \hline
			0.1 & Napravljen predložak i dodani opisi obrazaca uporabe.	& Barukčić & 31.10.2023. 		\\[3pt] \hline 
			0.2	& Napisani funkcionalni zahtjevi. \newline Dodani dijagrami obrazaca uporabe. & Jukić, \newline Samaržija & 2.11.2023. 	\\[3pt] \hline 
			0.3 & Dodan opis tablica i dijagram baze podataka. & Topolovec & 05.11.2023. \\[3pt] \hline 
			0.4 & Dodan jedan sekvencijski dijagram i opis. & Barić, Đunđek & 08.11.2023. \\[3pt] \hline 
			0.5 & Dodana arhitektura i dizajn sustava.\newline Dodani ostali zahtjevi. & Topolovec, Barić & 11.11.2023. \\[3pt] \hline 
			0.6 & Dodan opis projektnog zadatka. & Božić, \newline Barukčić & 13.11.2023. \\[3pt] \hline 
			0.7 & Dodan opis obrazaca uporabe 5 i\newline ažuriran dijagram obrazaca uporabe. & Đunđek & 13.11.2023. \\[3pt] \hline 
			0.8 & Dodan drugi sekvencijski dijagram i opis & Barić & 13.11.2023. \\[3pt] \hline 
			0.9 & Dodan dijagram razreda & Topolevec, Đunđek & 13.11.2023. \\[3pt] \hline 
			0.9.1 & Dodan opis dijagrama razreda & Barić, \newline Samaržija & 14.11.2023. \\[3pt] \hline 
			\textbf{1.0} & Verzija samo s bitnim dijelovima za 1. ciklus & Barukčić & 17.11.2023. \\[3pt] \hline 
			1.1 & Korištene tehnologije i alati & Barukčić & 30.12.2023. \\[3pt] \hline 
			1.2 & Upute za puštanje u pogon & Barukčić & 3.1.2024. \\[3pt] \hline 
			%1.3 & Popravljeni dijagrami obrazaca uporabe & * & 15.09.2013. \\[3pt] \hline 
			%1.5 & Generalna revizija strukture dokumenta & * & 19.09.2013. \\[3pt] \hline 
			%1.5.1 & Manja revizija (dijagram razmještaja) & * & 20.09.2013. \\[3pt] \hline 
			%\textbf{2.0} & Konačni tekst predloška dokumentacije  & * & 28.09.2013. \\[3pt] \hline 
			%&  &  & \\[3pt] \hline	
		\end{longtblr}
	
	
		%\textit{Moraju postojati glavne revizije dokumenata 1.0 i 2.0 na kraju prvog i drugog ciklusa. Između tih revizija mogu postojati manje revizije već prema tome kako se dokument bude nadopunjavao. Očekuje se da nakon svake značajnije promjene (dodatka, izmjene, uklanjanja dijelova teksta i popratnih grafičkih sadržaja) dokumenta se to zabilježi kao revizija. Npr., revizije unutar prvog ciklusa će imati oznake 0.1, 0.2, …, 0.9, 0.10, 0.11.. sve do konačne revizije prvog ciklusa 1.0. U drugom ciklusu se nastavlja s revizijama 1.1, 1.2, itd.}