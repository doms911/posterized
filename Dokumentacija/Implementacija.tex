\chapter{Implementacija i korisničko sučelje}
		
		
		\section{Korištene tehnologije i alati}
			
			Tim uspješno komunicira putem aplikacija Discord i WhatsApp, omogućavajući efikasnu i brzu razmjenu informacija. 
			
			UML dijagrami napisani su u okruženju AstahUML. Koristimo ih za vizualizaciju i analizu softverskog rješenja, što pomaže boljem razumijevanju arhitekture i funkcionalnosti sustava.
			
			U procesu razvoja koristimo Git za upravljanje verzijama koda, s udaljenim repozitorijem na GitHubu. Ovo omogućava timsku suradnju, praćenje promjena i povratak na prethodne verzije koda po potrebi. Korištena razvojna okruženja su IntelliJ za programski jezik Java i WebStorm za p. jezik JavaScript, čime smo osigurali snažne alate za razvoj i debugiranje aplikacije. Što se tiče samog softverskog rješenja, frontend aplikacija je napisana u JavaScriptu, koristeći biblioteku ReactJS i Node.js za poslužitelja web aplikacije. Backend je napisan u Java Spring framework-u, time je pružena stabilnost serverskom dijelu.
			
			Za registraciju koristimo Google reCAPTCHA API kako bi povećavali sigurnost aplikacije. Sve slike koje koristimo skladištimo na Firebase Cloud platformi, pružajući pouzdanu infrastrukturu za upravljanje multimedijskim sadržajem. 
			U fazi razvoja koristimo H2 bazu podataka, dok aplikacija u pogonu koristi udaljenu PostgreSQL instancu pruženu od strane Render platforme. Pogon aplikacije vršimo putem Render platforme, što omogućava jednostavan i efikasan proces.
			
			Za pisanje dokumentacije koristimo TexStudio okruženje, koje podržava LaTeX jezik. Ova kombinacija omogućava strukturirano i profesionalno dokumentiranje rješenja. 
			
			Sve ove tehnologije i alati zajedno čine tim sposobnim za efikasan razvoj, održavanje i dokumentiranje softverskog rješenja.
			
			\eject 
		
	
		\section{Ispitivanje programskog rješenja}
			
	Testiranje programskog riješenja smo proveli na dva        načina:
        \begin{itemize}
            \item Ispitivanje komponenti
            \item Ispitivanje sustava
        \end{itemize}
        Ispitivanje komponenti smo ostvarili pomoću JUnit testova u programskom jeziku Java.\\
        Za ispitivanje sustava koristili smo Selenium IDE. Selenium IDE je jednostavan, baziran na web pregledniku alat za automatizaciju testiranja web aplikacija koji omogućuje snimanje, uređivanje i ponovno izvođenje testova izravno unutar web preglednika.\\
        Svaki dio sustava je ispitan, no zbog jednostavnosti u dokumentaciji ce biti prikazan samo dio ispitivanja.

    

			
		
	
			
			\subsection{Ispitivanje komponenti}
			\textitSvi 
        Testni slučajevi analiziraju tok programa i međusobnu interakciju njegovih komponenti, poput broja poziva pojedine funkcije, pojavljivanja iznimaka, uspješnosti dohvaćanja željenih resursa, i slično.\\
        \\
        \textbf{Testni slučajevi} \\
        \begin{enumerate}
    \item Registracija novog korisnika\\
    Testira poziv repozitoriju.
    \begin{figure}[H]
        \includegraphics[scale=0.30]{slike/deploy/backTest1.png}
        \centering
        \label{fig:promjene1}
    \end{figure}

    \item Pokušaj registracije null (bez podataka) korisnika\\
    Testiranje pozivanje funkcije createAutor s null osobom.
    \begin{figure}[H]
        \includegraphics[scale=0.30]{slike/deploy/backTest2.png}
        \centering
        \label{fig:promjene2}
    \end{figure}

    \item Pokušaj registracije korisnika bez emaila
    \begin{figure}[H]
        \includegraphics[scale=0.30]{slike/deploy/backTest3.png}
        \centering
        \label{fig:promjene3}
    \end{figure}

    % Ovdje postavljamo brojač na 4
    \setcounter{enumi}{3}

    \item Pokušaj dodavanja mjesta koje se već nalazi u bazi
    \begin{figure}[H]
        \includegraphics[scale=0.30]{slike/deploy/backTest4.png}
        \centering
        \label{fig:promjene4}
    \end{figure}

    \item Pokušaj stvaranja konferencije koja se već nalazi u bazi
    \begin{figure}[H]
        \includegraphics[scale=0.30]{slike/deploy/backTest5.png}
        \centering
        \label{fig:promjene5}
    \end{figure}
    \\
    \item Testiranje poziva funkcije za poziv token pri resetiranju lozinke.
    \begin{figure}[H]
        \includegraphics[scale=0.30]{slike/deploy/backTest6.png}
        \centering
        \label{fig:promjene6}
    \end{figure}
\end{enumerate}
\textbf{Rezultati testova} \\
\begin{figure}[H]
        \includegraphics[scale=0.30]{slike/deploy/backTestResults.png}
        \centering
        \label{fig:promjene6}
    \end{figure}
			
			
			
			\subsection{Ispitivanje sustava}
			
			 \textit
                Ispitivanje čitavog sustava provedeno je pomoću alata Selenium IDE korištenjem radnog okvira Selenium. Ispitani su redovni slučajevi s realnim ulaznim parametrima, rubni uvjeti i poziv funkcionalnosti koja nije implementirana/izaziva pogresku pomoću čega je vidljivo na koji način sustav reagira kada nešto nije u potpunosti ostvareno.\\
                \\
                \textbf{Tijek izvodenja:} \\
        \begin{enumerate}
            \item Klik na Selenium IDE u Google Chrome Browseru
            \item Klik na “Add new test”
            \item Upisivanje početne adrese
            \item Pokretanje testa koji snima sve događaje
            \item Spremanje testa
            \item Pokretanje testa koji reproducira snimljenje događaje
        \end{enumerate}
		 
  Koristimo Selenium za automatizirano testiranje, zbog čega zaobilazimo CAPTCHA mehanizam u našem testnom okruženju. CAPTCHA je dizajnirana da spriječi automatizaciju, stoga njeno zaobilaženje omogućuje kontinuirano i učinkovito testiranje bez potrebe za ručnim unosom.\\
  \\
  
   \textbf{Testni slučajevi} \\
        \begin{enumerate}
    \item Uspješna odjava korisnika\\
        Odjavljujemo prijavljenog korisnika.
    \begin{figure}[H]
        \includegraphics[scale=0.30]{slike/deploy/fTest2a.png}
        \centering
        \label{fig:promjene2}
    \end{figure}
    \begin{figure}[H]
        \includegraphics[scale=0.30]{slike/deploy/fTest2b.png}
        \centering
        \label{fig:promjene1}
    \end{figure}
    

    \item Neuspješna prijava korisnika\\
    Prijavljujemo se kao vec registrirani korisnik sa netočnim podatcima(lozinkom).
    \begin{figure}[H]
        \includegraphics[scale=0.30]{slike/deploy/fTest1a.png}
        \centering
        \label{fig:promjene2}
    \end{figure}
    \begin{figure}[H]
        \includegraphics[scale=0.30]{slike/deploy/fTest1b.png}
        \centering
        \label{fig:promjene1}
    \end{figure}

    \item Neuspješna registracija korisnika\\
    Pri registraciji nije unesen email te se zato registracija ne izvršava pritiskom na gumb registracije.
    \begin{figure}[H]
        \includegraphics[scale=0.30]{slike/deploy/fTest3a.png}
        \centering
        \label{fig:promjene2}
    \end{figure}
    \begin{figure}[H]
        \includegraphics[scale=0.30]{slike/deploy/fTest3b.png}
        \centering
        \label{fig:promjene1}
    \end{figure}

    % Ovdje postavljamo brojač na 4
    \setcounter{enumi}{3}

    \item Uspješna redirekcija\\
     Uspješno mjenjamo stranice iz početne stranice na stranicu registracije. Nakon toga prebacujemo se na stranicu prijave.
    \begin{figure}[H]
        \includegraphics[scale=0.30]{slike/deploy/fTest4a.png}
        \centering
        \label{fig:promjene2}
    \end{figure}
    \begin{figure}[H]
        \includegraphics[scale=0.30]{slike/deploy/fTest4b.png}
        \centering
        \label{fig:promjene1}
    \end{figure}

    

  
		
		\eject
		\section{Dijagram razmještaja}

			Dijagrami razmještaja pružaju uvid u organizaciju hardverskih komponenti i programske podrške koja se koristi u implementaciji sustava unutar njegovog radnog okruženja. Na jednom od poslužiteljskih računala smješten je web poslužitelj, dok se na drugom nalazi poslužitelj baze podataka. Klijenti koriste web preglednik kako bi pristupili web aplikaciji. Sustav je temeljen na klijent-poslužitelj arhitekturi, pri čemu se komunikacija između računala korisnika i poslužitelja odvija putem HTTP veze.
			
			Web poslužitelj odgovoran je za pružanje web stranica i servisa korisnicima, dok poslužitelj baze podataka čuva, upravlja i omogućava pristup podacima. Korištenjem web preglednika, korisnici ostvaruju pristup aplikaciji koja se nalazi na sustavu, a komunikacija se odvija preko standardne HTTP veze, osiguravajući efikasnu interakciju između klijenta i poslužitelja.
			
			\begin{figure}[H]
				\includegraphics[scale=0.35]{dijagrami/dijagram_razmjestaja.png} %veličina slike u odnosu na originalnu datoteku i pozicija slike
				\centering
				\caption{Dijagram razmještaja}
				\label{fig:promjene}
			\end{figure}
			
			\eject 
		
		\section{Upute za puštanje u pogon}
		
		Aplikacija je puštena u pogon koristeći alat Render. Koristeći korisničko sučelje koje Render pruža treba kreirati instancu baze podataka, instancu za web servis backenda te instancu za web servis frontenda aplikacije. Prvo treba kreirati bazu podataka PostgreSQL.
			
			\begin{figure}[H]
				\includegraphics[scale=0.3]{slike/deploy/database1.png} %veličina slike u odnosu na originalnu datoteku i pozicija slike
				\centering
				\caption{Izbornik za stvaranje nove baze podataka}
				\label{fig:promjene}
			\end{figure}
			
			Potrebno je upisati ime baze, odabrati regiju poslužitelja instance i označiti tip instance koji će se koristiti.
			\begin{figure}[H]
				\includegraphics[scale=0.3]{slike/deploy/database2.png} %veličina slike u odnosu na originalnu datoteku i pozicija slike
				\centering
				\caption{Stvaranje nove baze podataka}
				\label{fig:promjene}
			\end{figure}
			Nakon ovih koraka treba pokrenuti stvaranje pritiskom na tipku "Create".
			
			Pokazuje se prozor u kojemu su navedeni osnovni podatci o bazi kao ime, verzija, regija, prostor za pohranu, API ključ, itd.
			\begin{figure}[H]
				\includegraphics[scale=0.3]{slike/deploy/database3.png} %veličina slike u odnosu na originalnu datoteku i pozicija slike
				\centering
				\caption{Osnovni podatci o statusu baze podataka}
				\label{fig:promjene}
			\end{figure}
			
			Kad je baza uspješno kreirana, potrebno je uzeti podatke koje treba dati backendu aplikacije. Ti podatci se nalaze u poljima \textit{Hostname}, \textit{Port}, \textit{Database}, \textit{Username}, \textit{Password} i \textit{External Database URL}.
			\begin{figure}[H]
				\includegraphics[scale=0.4]{slike/deploy/database4.png} %veličina slike u odnosu na originalnu datoteku i pozicija slike
				\centering
				\caption{Podatci za pristupanje bazi podataka}
				\label{fig:promjene}
			\end{figure}
			
			\pagebreak
			 Za kreiranje backend web servisa potrebno je prvo napraviti pripremu pa automatsko kreiranje web servisa iz GitHub repozitorija. U projektu razvojnog okruženja treba dodati port servera, kontekstnu putanju, podatke za pristup bazi upisati u označeni prozor te izbrisati aktivni dev profil.
			\begin{figure}[H]
				\includegraphics[scale=0.4]{slike/deploy/backend8.png} %veličina slike u odnosu na originalnu datoteku i pozicija slike
				\centering
				\caption{Datoteka \textit{application.properties}}
				\label{fig:promjene}
			\end{figure}
			U projekt je potrebno ubaciti odgovarajući \textit{Dockerfile} na putanju \textit{./posterized-backend/docker/maven} koja je prikazana stablom.
			\begin{figure}[H]
				\includegraphics[scale=0.8]{slike/deploy/backend7.png} %veličina slike u odnosu na originalnu datoteku i pozicija slike
				\centering
				\caption{Stablo u kojem se nalazi \textit{Dockerfile}}
				\label{fig:promjene}
			\end{figure}
			
			\pagebreak
			Kreiranje web servisa se pokreće iz Renderovog izbornika.
			\begin{figure}[H]
				\includegraphics[scale=0.4]{slike/deploy/backend00.png} %veličina slike u odnosu na originalnu datoteku i pozicija slike
				\centering
				\caption{Izbornik za kreiranje web servisa}
				\label{fig:promjene}
			\end{figure}
			Potrebno je povezati GitHub repozitorij s Renderom. Označiti "Build and deploy from a Git repository" i kliknuti "Next".
			\begin{figure}[H]
				\includegraphics[scale=0.4]{slike/deploy/backend1.png} %veličina slike u odnosu na originalnu datoteku i pozicija slike
				\centering
				\caption{Kreiranje web servisa - povezivanje s udaljenim repozitorijem}
				\label{fig:promjene}
			\end{figure}
			\pagebreak
			Kliknuti "Connect GitHub", prijaviti se u račun, odobriti autorizaciju i kliknuti "Connect".
			\begin{figure}[H]
				\includegraphics[scale=0.25]{slike/deploy/backend2.png} %veličina slike u odnosu na originalnu datoteku i pozicija slike
				\centering
				\caption{Povezivanje s repozitorijem}
				\label{fig:promjene}
			\end{figure}
			
			\begin{figure}[H]
				\includegraphics[scale=0.3]{slike/deploy/backend3.png} %veličina slike u odnosu na originalnu datoteku i pozicija slike
				\centering
				\caption{Pronađeni repozitorij}
				\label{fig:promjene}
			\end{figure}
			Upisati ime servisa i odabrati regiju koja je najbliža našoj lokaciji.  Pod \textit{Branch} odabrati master i upisati ime mape projekta. Odabrati tip instance koji će se koristiti.
			\begin{figure}[H]
				\includegraphics[scale=0.3]{slike/deploy/backend4.png} %veličina slike u odnosu na originalnu datoteku i pozicija slike
				\centering
				\caption{Kreiranje web servisa - upisivanje osnovnih podataka}
				\label{fig:promjene}
			\end{figure}
			
			Otvoriti padajući izbornik "Advanced" i upisati putanju gdje se nalazi \textit{Dockerfile}, zatim kliknuti "Create Web Service".
			\begin{figure}[H]
				\includegraphics[scale=0.35]{slike/deploy/backend5.png} %veličina slike u odnosu na originalnu datoteku i pozicija slike
				\centering
				\caption{Navođenje \textit{Dockerfile} putanje}
				\label{fig:promjene}
			\end{figure}
			
			Web servis backenda je uspješno kreiran i pušten u pogon.
			\begin{figure}[H]
				\includegraphics[scale=0.4]{slike/deploy/backend6.png} %veličina slike u odnosu na originalnu datoteku i pozicija slike
				\centering
				\caption{Logovi stvaranja i status web servisa}
				\label{fig:promjene}
			\end{figure}
			
			% za frontend
			\pagebreak
			Prije kreiranja web servisa za frontend, potrebno je prvo pripremiti projekt frontenda za puštanje u pogon.
			Treba napisati \textit{app.js} datoteku kao na slici. Ova datoteka je dio Node.js-a, odnosno Express frameworka koji je odgovoran u izgradnji RESTful API-ja za jednostrane, višestrane ili hibridne aplikacije.
			
			\begin{figure}[H]
				\includegraphics[scale=0.4]{slike/deploy/frontend5.png} %veličina slike u odnosu na originalnu datoteku i pozicija slike
				\centering
				\caption{Datoteka \textit{app.js}}
				\label{fig:promjene}
			\end{figure}
			
			Napisati \textit{setupProxy.js} za uspostavljanje proxyja između razvojnog okružja i API-ja koji služi izbjegavanju CORS grešaka što pridonosi lakšem razvoju aplikacije.
			\begin{figure}[H]
				\includegraphics[scale=0.4]{slike/deploy/frontend6.png} %veličina slike u odnosu na originalnu datoteku i pozicija slike
				\centering
				\caption{Datoteka \textit{setupProxy.js}}
				\label{fig:promjene}
			\end{figure}
			
			\pagebreak
			Potrebno je izmijeniti datoteku \textit{package.lock} kao na slici u označenom prozoru. 
			\begin{figure}[H]
				\includegraphics[scale=0.4]{slike/deploy/frontend7.png} %veličina slike u odnosu na originalnu datoteku i pozicija slike
				\centering
				\caption{Datoteka \textit{package.lock}}
				\label{fig:promjene}
			\end{figure}
			
			Kreiranje web servisa za frontend slično je kao za backend uz manje razlike. Započinje se u izborniku kao što je navedeno za backend.
			
			U postupku kreiranja web servisa u označene prozore treba upisati ime, kao \textit{Branch} odabrati master te ispod upisati ime mape projekta. Za \textit{Runtime} odabrati Node i označiti tip instance koji će se koristiti.
			\begin{figure}[H]
				\includegraphics[scale=0.3]{slike/deploy/frontend2.png} %veličina slike u odnosu na originalnu datoteku i pozicija slike
				\centering
				\caption{Kreiranje web servisa za frontend}
				\label{fig:promjene}
			\end{figure}
			
			\pagebreak
			Ispod \textit{Runtime} pojavili su se linije za naredbe build i start te ih je potrebno izmijeniti kao na slici.
			\begin{figure}[H]
				\includegraphics[scale=0.3]{slike/deploy/frontend3.png} %veličina slike u odnosu na originalnu datoteku i pozicija slike
				\centering
				\caption{Naredbe za build i start}
				\label{fig:promjene}
			\end{figure}
			
			Odabrati "Environment Variables" i kreirati novu varijablu okruženja za adresu backenda koja je dostupna u "Dashboard" na Renderu. Jedinstvena adresa se dodjeljuje svakom kreiranom web servisu. Zatim kliknuti "Create Web Service" i pričekati "build" i "deploy" procese.
			\begin{figure}[H]
				\includegraphics[scale=0.3]{slike/deploy/frontend4.png} %veličina slike u odnosu na originalnu datoteku i pozicija slike
				\centering
				\caption{Varijable okruženja}
				\label{fig:promjene}
			\end{figure}
			
			Kada status servisa postane "Live", kliknuti na "Dashboard". Ako je sve uspješno postavljeno, instance će biti prikazane kao na slici.
			\begin{figure}[H]
				\includegraphics[scale=0.3]{slike/deploy/success.png} %veličina slike u odnosu na originalnu datoteku i pozicija slike
				\centering
				\caption{Uspostavljene instance}
				\label{fig:promjene}
			\end{figure} 
			
			Aplikacija je puštena u pogon i spremna za uporabu. Pristupa se preko adrese koja je navedena za instancu frontenda. 
			
			\href{https://posterized.onrender.com}{https://posterized.onrender.com}
			
			\eject 