\chapter{Zaključak i budući rad}
		 
		 Zadatak našeg tima bio je razviti web aplikaciju za digitalne postere namijenjene prezentacijama na konferencijama. Cilj projekta uspješno je postignut kroz dvije faze.
		 
		 U prvoj fazi, formirali smo tim i međusobno se upoznali, raspravljajući o zadatku, mogućim idejama i planu razvoja projekta. Tim je bio podijeljen na backend i frontend podtimove, dok je voditelj projekta bio zadužen za dokumentaciju i organizaciju. Postignute su osnovne funkcionalnosti, uključujući početnu stranicu, registraciju i prijavu korisnika. Također, izrađen je značajan dio dokumentacije koji uključuje uvod, obrasce uporabe, prateće UML dijagrame, funkcionalne i nefunkcionalne zahtjeve, arhitekturu sustava te ostale UML dijagrame. Članovi tima su znatno pridonijeli pisanju dokumentacije i organizaciji projekta.
		 
		 U drugoj fazi, dovršeni su preostali dijelovi dokumentacije i funkcionalnosti aplikacije. Backend i frontend timovi su usko surađivali i uspješno dovršili razvoj aplikacije. Dokumentacija je proširena uputama za puštanje aplikacije u pogon, dijagramom razmještaja, dijagramom aktivnosti, dijagramom stanja, dijagramom komponenata, revizijom dijagrama razreda i zaključkom.
		 
		 Komunikacija unutar tima odvijala se putem serverskih kanala na platformi Discord, osiguravajući da su svi članovi jednako informirani i uključeni u projekt. Zadaci su bili individualno raspoređeni među članovima tima, s mogućnošću da svaki član preuzme inicijativu i uključi se u bilo koji segment projekta. Korištene tehnologije uključuju Java Spring, JavaScript, React biblioteku i platformu Render za puštanje aplikacije u pogon.
		 
		 Kao moguća proširenja aplikacije, razmatramo dodavanje UX/UI dizajnera za unaprjeđenje korisničkog sučelja, razvoj mobilne aplikacije te redizajn aplikacije kako bi se omogućilo svakoj dvorani, hotelu ili bilo kojoj nekretnini s društvenom namjenom da ima vlastitu verziju aplikacije, slično platformi Moodle koja se koristi u akademskim ustanovama.
		
		\eject 